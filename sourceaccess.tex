%
% Copyright � 2014 Peeter Joot.  All Rights Reserved.
% Licenced as described in the file LICENSE under the root directory of this GIT repository.
%
Here is an outline of how to get access to the sources for this document

\begin{itemize}
\item Set up a github userid and ssh key for yourself.  Instructions for that here can be found on \href{http://help.github.com/win-set-up-git/}{github}.

Even if you do not want to use this for my source repository, if you are not using version control software for your own document text files I \textit{highly} recommend you do so.

\item Install a git client on your machine.  I use either linux or a \href{http://www.cygwin.com/}{Windows cygwin environment} (free implementation of gnu Unix command line utilities.  Any other Unix would work provided you have both gnu-make, perl and a recent latex distribution.  I use MikTex on Windows, and texlive on Linux.

\item Get your self a copy of the source repository that contains the sources.  Here is an example of how to do so

\lstinputlisting[language=bash]{gitclone.ksh}

If there are mathematica notebooks associated with this document, they can be found with the sources.  You can execute any of those with the \href{http://www.wolfram.com/cdf-player/}{free Wolfram CDF} player once you install it.

\item Build yourself a copy of the pdf.  For example, after cloning the repository as above

\lstinputlisting[language=bash]{sampleBuildMake.ksh}

The first time you do this with MikTex, you will probably take a hit to install a number of packages.  If running on a non-Linux Unix platform, use gnu-make explicitly.  The document structure can be explored starting from main.tex.

\item After running the git clone command, an invokation of:

\lstinputlisting[language=bash]{gitpull.ksh}

from anywhere in the git created phyLatex directory, will get you a more recent version of the source if there have been updates.
\end{itemize}
